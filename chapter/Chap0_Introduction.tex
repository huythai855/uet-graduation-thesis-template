\clearpage
\phantomsection

\addcontentsline{toc}{chapter}{{Mở đầu}}
\chapter*{Mở đầu}
\noindent{\Large \textbf{Lý do chọn đề tài}}


Bối cảnh (ví dụ: sự phát triển của trí tuệ nhân tạo?)

Các hạn chế của các nền tảng hiện tại?

Sự cần thiết của một hệ thống hỗ trợ triển khai mô hình học máy trên hạ tầng tại chỗ?


\noindent{\Large \textbf{Đóng góp của đề tài}}

Trong khóa luận này, tôi tập trung nghiên cứu, thiết kế..... .  Hệ thống gồm các tính năng nổi bật:

\renewcommand{\labelitemi}{$-$}
\begin{itemize}
	\item Tính năng 1
	\item Tính năng 2
	\item ....
\end{itemize}
\vspace{0.3cm}

\noindent{\Large \textbf{Bố cục của khóa luận}}
% \vspace{0.5cm}

Nội dung của khóa luận được trình bày như sau:

\renewcommand{\labelitemi}{$-$}
\begin{itemize}
	\item \textit{Mở đầu}: Trình bày mục đích, nội dung và bố cục của khóa luận.
	\item \textit{Chương 1. Tên chương 1}: Trong chương này, khóa luận tốt nghiệp sẽ giới thiệu ....
	\item \textit{Chương 2. Tên chương 2}: Trong chương này, khóa luận tốt nghiệp sẽ giới thiệu ....
	\item \textit{Chương 3. Tên chương 3}: Trong chương này, khóa luận tốt nghiệp sẽ giới thiệu ....
	\item \textit{Kết luận}: Đưa ra kết luận về việc xây dựng hệ thống ....
\end{itemize} 